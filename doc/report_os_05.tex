\documentclass[12pt]{article}

\usepackage{fullpage}
\usepackage{multicol,multirow}
\usepackage{tabularx}
\usepackage{ulem}
\usepackage[utf8]{inputenc}
\usepackage[russian]{babel}
\usepackage{minted}

\usepackage{color} %% это для отображения цвета в коде
\usepackage{listings} %% собственно, это и есть пакет listings

\lstset{ %
  language=C,                 % выбор языка для подсветки (здесь это С)
  basicstyle=\small\sffamily, % размер и начертание шрифта для подсветки кода
  numbers=left,               % где поставить нумерацию строк (слева\справа)
  %numberstyle=\tiny,           % размер шрифта для номеров строк
  stepnumber=1,                   % размер шага между двумя номерами строк
  numbersep=5pt,                % как далеко отстоят номера строк от подсвечиваемого кода
  backgroundcolor=\color{white}, % цвет фона подсветки - используем \usepackage{color}
  showspaces=false,            % показывать или нет пробелы специальными отступами
  showstringspaces=false,      % показывать или нет пробелы в строках
  showtabs=false,             % показывать или нет табуляцию в строках
  frame=single,              % рисовать рамку вокруг кода
  tabsize=2,                 % размер табуляции по умолчанию равен 2 пробелам
  captionpos=t,              % позиция заголовка вверху [t] или внизу [b] 
  breaklines=true,           % автоматически переносить строки (да\нет)
  breakatwhitespace=false, % переносить строки только если есть пробел
  escapeinside={\%*}{*)}   % если нужно добавить комментарии в коде
}

\begin{document}
\begin{titlepage}
  \large
  \begin{center} 
    
      Московский Авиационный Интститут \\
      (Национальный Исследовательский Университет) \\
      Факультет информационных технологий и прикладной математики \\
      Кафедра вычислительной математики и программирования \\
      \vfill\vfill
      \textbf{
        { Лабораторная работа №5 по курсу} \\ 
        <<Операционные системы>> \\
        \bigskip
            {Создание и использование динамических библиотек } \\
    } 
  \end{center}
  \vfill

  \begin{flushright}

    Студент:  {Артемьев Дмитрий Иванович}

    Группа: {М8О-206Б-18}

    Вариант: {23}
    
    Преподаватель: {Соколов Андрей Алексеевич}

    Оценка: $\rule{3cm}{0.15mm}$

    Дата: $\rule{3cm}{0.15mm}$
    
    Подпись: $\rule{3cm}{0.15mm}$

  \end{flushright}
  \vfill
  \begin{center}
    Москва, 2019
  \end{center}
  
\end{titlepage}

\subsection*{Условие}

Требуется создать динамическую библиотеку, которая реализует определенный функционал.

Далее использовать данную библиотеку 2-мя способами:

1. Во время компиляции (на этапе «линковки»/linking)

2. Во время исполнения программы, подгрузив библиотеку в память с помощью системных
вызовов

В конечном итоге, программа должна состоять из следующих частей:

• Динамическая библиотека, реализующая заданных вариантом интерфейс;

• Тестовая программа, которая используют библиотеку, используя знания полученные на
этапе компиляции;

• Тестовая программа, которая использует библиотеку, используя только местоположение
динамической библиотеки и ее интерфейс.

Провести анализ между обоими типами использования библиотеки.

3. Работа с деком, целочисленный 32-битный. 

\subsection*{Описание программы}

Код программы находится в файлах before\_main.c, at\_time\_main.c, deque.c, deque.h.

\subsection*{Анализ типов динамических библиотек}

Динамические библиотеки, хоть и замедляют загрузку программы, обладают существенными преимуществами перед статическими: нет необходимости копировать библиотеку для каждой отдельной программы, также в одном варианте возможно применять изменения библиотеки в уже запущенной программе. Существует два типа динамических библиотек. Первый из них предполагает линковку во время компиляции. При этом становится невозможно применять изменения, происходящие в библиотеке для запущенной программы. Эту проблему решает второй тип: библиотека, подгружаемая программой во время исполнения с помощью системных вызовов. Таким образом использование динамических библиотек, подключаемых на этапе выполнения программы является более гибким решением. Однако при этом нужно больше задумываться о безопасности. 

\subsection*{Выводы}

Я научился создавать и использовать динамические библиотеки в операционной системе Linux. 
\pagebreak

\vfill

\subsection*{Исходный код}

{\Huge before\_main.c}
\inputminted
    {C}{../src/before_main.c}
    \pagebreak    
    
{\Huge at\_time\_main.c}
\inputminted
    {C}{../src/at_time_main.c}
    \pagebreak    

{\Huge deque.c}
\inputminted
    {C}{../src/deque.c}
    \pagebreak    

{\Huge main.h}
\inputminted
    {C}{../src/deque.h}
    \pagebreak    

\end{document}
